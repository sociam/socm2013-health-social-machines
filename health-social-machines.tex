% This is "sig-alternate.tex" V2.0 May 2012
% This file should be compiled with V2.5 of "sig-alternate.cls" May 2012
%
% This example file demonstrates the use of the 'sig-alternate.cls'
% V2.5 LaTeX2e document class file. It is for those submitting
% articles to ACM Conference Proceedings WHO DO NOT WISH TO
% STRICTLY ADHERE TO THE SIGS (PUBS-BOARD-ENDORSED) STYLE.
% The 'sig-alternate.cls' file will produce a similar-looking,
% albeit, 'tighter' paper resulting in, invariably, fewer pages.
%
% ----------------------------------------------------------------------------------------------------------------
% This .tex file (and associated .cls V2.5) produces:
%       1) The Permission Statement
%       2) The Conference (location) Info information
%       3) The Copyright Line with ACM data
%       4) NO page numbers
%
% as against the acm_proc_article-sp.cls file which
% DOES NOT produce 1) thru' 3) above.
%
% Using 'sig-alternate.cls' you have control, however, from within
% the source .tex file, over both the CopyrightYear
% (defaulted to 200X) and the ACM Copyright Data
% (defaulted to X-XXXXX-XX-X/XX/XX).
% e.g.
% \CopyrightYear{2007} will cause 2007 to appear in the copyright line.
% \crdata{0-12345-67-8/90/12} will cause 0-12345-67-8/90/12 to appear in the copyright line.
%
% ---------------------------------------------------------------------------------------------------------------
% This .tex source is an example which *does* use
% the .bib file (from which the .bbl file % is produced).
% REMEMBER HOWEVER: After having produced the .bbl file,
% and prior to final submission, you *NEED* to 'insert'
% your .bbl file into your source .tex file so as to provide
% ONE 'self-contained' source file.
%
% ================= IF YOU HAVE QUESTIONS =======================
% Questions regarding the SIGS styles, SIGS policies and
% procedures, Conferences etc. should be sent to
% Adrienne Griscti (griscti@acm.org)
%
% Technical questions _only_ to
% Gerald Murray (murray@hq.acm.org)
% ===============================================================
%
% For tracking purposes - this is V2.0 - May 2012

\documentclass{sig-alternate}

\usepackage{url}

\begin{document}
%
% --- Author Metadata here ---
\conferenceinfo{Workshop on the theory and practice of social machines @ WWW2013}{2013, Rio de Janeiro, Brazil}
%\CopyrightYear{2007} % Allows default copyright year (20XX) to be over-ridden - IF NEED BE.
%\crdata{0-12345-67-8/90/01}  % Allows default copyright data (0-89791-88-6/97/05) to be over-ridden - IF NEED BE.
% --- End of Author Metadata ---

\title{``The Crowd Keeps Me in Shape'': Psychology and the Past, Present and Future of Health Social Machines}
%
% You need the command \numberofauthors to handle the 'placement
% and alignment' of the authors beneath the title.
%
% For aesthetic reasons, we recommend 'three authors at a time'
% i.e. three 'name/affiliation blocks' be placed beneath the title.
%
% NOTE: You are NOT restricted in how many 'rows' of
% "name/affiliations" may appear. We just ask that you restrict
% the number of 'columns' to three.
%
% Because of the available 'opening page real-estate'
% we ask you to refrain from putting more than six authors
% (two rows with three columns) beneath the article title.
% More than six makes the first-page appear very cluttered indeed.
%
% Use the \alignauthor commands to handle the names
% and affiliations for an 'aesthetic maximum' of six authors.
% Add names, affiliations, addresses for
% the seventh etc. author(s) as the argument for the
% \additionalauthors command.
% These 'additional authors' will be output/set for you
% without further effort on your part as the last section in
% the body of your article BEFORE References or any Appendices.

\numberofauthors{1} %  in this sample file, there are a *total*
% of EIGHT authors. SIX appear on the 'first-page' (for formatting
% reasons) and the remaining two appear in the \additionalauthors section.
%
\author{
% You can go ahead and credit any number of authors here,
% e.g. one 'row of three' or two rows (consisting of one row of three
% and a second row of one, two or three).
%
% The command \alignauthor (no curly braces needed) should
% precede each author name, affiliation/snail-mail address and
% e-mail address. Additionally, tag each line of
% affiliation/address with \affaddr, and tag the
% e-mail address with \email.
%
% 1st. author
\alignauthor
Authors\\
       \affaddr{Web and Internet Science Group}\\
       \affaddr{University of Southampton}\\
       \affaddr{Southampton, UK}\\
       \email{\{a,b,c,d,wh,nrs\}@ecs.soton.ac.uk}
}
% There's nothing stopping you putting the seventh, eighth, etc.
% author on the opening page (as the 'third row') but we ask,
% for aesthetic reasons that you place these 'additional authors'
% in the \additional authors block, viz.
%\additionalauthors{Additional authors: John Smith (The Th{\o}rv{\"a}ld Group,
%email: {\texttt{jsmith@affiliation.org}}) and Julius P.~Kumquat
%(The Kumquat Consortium, email: {\texttt{jpkumquat@consortium.net}}).}
%\date{30 July 1999}
% Just remember to make sure that the TOTAL number of authors
% is the number that will appear on the first page PLUS the
% number that will appear in the \additionalauthors section.

\maketitle
\begin{abstract}

Health social machines ...

\end{abstract}

% A category with the (minimum) three required fields
%\category{H.4}{Information Systems Applications}{Miscellaneous}
%A category including the fourth, optional field follows...
%\category{D.2.8}{Software Engineering}{Metrics}[complexity measures, performance measures]

%\terms{Theory}

%\keywords{ACM proceedings, \LaTeX, text tagging}

\section{Introduction}

Health and well-being have long been used as visible indicators of
human technological progress, as advances in healthcare and medicine
are invariably reflected in quantifiable values such as increases in
average lifespan, reduction of disease and suffering, and shortening
of time needed to recover from illness and injury.  As such, it is
natural to ask how and whether the Internet and the Web, two of the
most significant inventions in recent human history, have and may
effect these measures.

In this position paper, we examine a specific type technology, Web and
Internet-enabled system, the emerging class of \emph{health social
  machines}, to see how they have, thus far, brought about advances
and techniques by which they can improve health and wellbeing from the
scale of the individual up through greater segments of society.  For
the purposes of this paper, we define a health social machine to be
Web-, app- and sensor-based online community or site where people
communicate and interact, mediated and facilitated through a digital
moderation mechanism, to collectively solve health-related problems.

We first examine the emerging landscape of health-related social
machines, and identify a set of classes of social machines based on
their characteristics and goals.  We then examine the goals of these
classes, in detail the discussion with an examination of

%% The digital substrates of health social
%% machines

%% one of the foremost inventions of modern human
%% civilisation, has

%%  of considerable as a
%% potential application of one of the foremost technological inventions
%% of human civilization, the Internet and the Web,

%% Yet formulating the general challenges of preventative medicine,
%% wellness, disease management, behavioural medicine, sou

%% In this position paper, w



%% types of health social machines
%% wellness and prevention
%% disease management
%% behavioural health
%% medication adherence

\section{A Classificatory Analysis of Extant Health Social Machines}

We identified three main objectives of extant, health-related social
machines: \emph{behavioural intervention}, \emph{disease management},
and \emph{disease understanding/medical science}.

The first which we refer to as ``behavioural intervention social
machines'' are systems that seek to help individuals achieve health
goals by changing their behaviour.  Looking purely at Web and device
startups, the majority of systems in this space seem to be ``fitness
social machines'', \emph{preventative wellness} systems that seek to
help individuals increase their fitness by either increasing their
general activity levels, or by setting specific strength and fitness
goals. But other systems include \emph{disease management}, to help
seeking to increase people's levels of physical activity, for example,
in order to increase reduce likelihood of the later onset of
conditions associated


\begin{table}[htb]
\begin{center}
\begin{tabular}{|p{8cm}|}
\hline
{\bf Preventative wellness} \\
\hline

\hline
{\bf Disease management} \\
\hline
Alzconnected (Alzeheimer's patients)

\end{tabular}
\end{center}
\caption{Consolidated constructs of social machines.} \label{table:constructs}
\end{table}





\subsection{Social pressure and Motivation: Gym Memberships and Personal Trainers}

\subsection{Present: Channel factors, access, convenience}
\subsection{Present: Salience and reminders}
\subsection{Futures: Personalised Activity Diaries}
\subsection{Futures: Citizen-medicine}

\section{Acknowledgments}

This work is supported under SOCIAM: The Theory and Practice of Social
Machines.  The SOCIAM Project is funded by the UK Engineering and
Physical Sciences Research Council (EPSRC) under grant number
EP/J017728/1 and comprises the Universities of Southampton, Oxford and
Edinburgh.

%
% The following two commands are all you need in the
% initial runs of your .tex file to
% produce the bibliography for the citations in your paper.
\bibliographystyle{abbrv}
\bibliography{sigproc}  % sigproc.bib is the name of the Bibliography in this case
% You must have a proper ".bib" file
%  and remember to run:
% latex bibtex latex latex
% to resolve all references
%
% ACM needs 'a single self-contained file'!
%
%APPENDICES are optional


\balancecolumns % GM June 2007
% That's all folks!
\end{document}
