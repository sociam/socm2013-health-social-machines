% This is "sig-alternate.tex" V2.0 May 2012
% This file should be compiled with V2.5 of "sig-alternate.cls" May 2012
%
% This example file demonstrates the use of the 'sig-alternate.cls'
% V2.5 LaTeX2e document class file. It is for those submitting
% articles to ACM Conference Proceedings WHO DO NOT WISH TO
% STRICTLY ADHERE TO THE SIGS (PUBS-BOARD-ENDORSED) STYLE.
% The 'sig-alternate.cls' file will produce a similar-looking,
% albeit, 'tighter' paper resulting in, invariably, fewer pages.
%
% ----------------------------------------------------------------------------------------------------------------
% This .tex file (and associated .cls V2.5) produces:
%       1) The Permission Statement
%       2) The Conference (location) Info information
%       3) The Copyright Line with ACM data
%       4) NO page numbers
%
% as against the acm_proc_article-sp.cls file which
% DOES NOT produce 1) thru' 3) above.
%
% Using 'sig-alternate.cls' you have control, however, from within
% the source .tex file, over both the CopyrightYear
% (defaulted to 200X) and the ACM Copyright Data
% (defaulted to X-XXXXX-XX-X/XX/XX).
% e.g.
% \CopyrightYear{2007} will cause 2007 to appear in the copyright line.
% \crdata{0-12345-67-8/90/12} will cause 0-12345-67-8/90/12 to appear in the copyright line.
%
% ---------------------------------------------------------------------------------------------------------------
% This .tex source is an example which *does* use
% the .bib file (from which the .bbl file % is produced).
% REMEMBER HOWEVER: After having produced the .bbl file,
% and prior to final submission, you *NEED* to 'insert'
% your .bbl file into your source .tex file so as to provide
% ONE 'self-contained' source file.
%
% ================= IF YOU HAVE QUESTIONS =======================
% Questions regarding the SIGS styles, SIGS policies and
% procedures, Conferences etc. should be sent to
% Adrienne Griscti (griscti@acm.org)
%
% Technical questions _only_ to
% Gerald Murray (murray@hq.acm.org)
% ===============================================================
%
% For tracking purposes - this is V2.0 - May 2012

\documentclass{sig-alternate}

\usepackage{url}

\begin{document}
%
% --- Author Metadata here ---
\conferenceinfo{Workshop on the theory and practice of social machines @ WWW2013}{2013, Rio de Janeiro, Brazil}
%\CopyrightYear{2007} % Allows default copyright year (20XX) to be over-ridden - IF NEED BE.
%\crdata{0-12345-67-8/90/01}  % Allows default copyright data (0-89791-88-6/97/05) to be over-ridden - IF NEED BE.
% --- End of Author Metadata ---

\title{``The Crowd Keeps Me in Shape'': Psychology and the Past, Present and Future of Health Social Machines}
%
% You need the command \numberofauthors to handle the 'placement
% and alignment' of the authors beneath the title.
%
% For aesthetic reasons, we recommend 'three authors at a time'
% i.e. three 'name/affiliation blocks' be placed beneath the title.
%
% NOTE: You are NOT restricted in how many 'rows' of
% "name/affiliations" may appear. We just ask that you restrict
% the number of 'columns' to three.
%
% Because of the available 'opening page real-estate'
% we ask you to refrain from putting more than six authors
% (two rows with three columns) beneath the article title.
% More than six makes the first-page appear very cluttered indeed.
%
% Use the \alignauthor commands to handle the names
% and affiliations for an 'aesthetic maximum' of six authors.
% Add names, affiliations, addresses for
% the seventh etc. author(s) as the argument for the
% \additionalauthors command.
% These 'additional authors' will be output/set for you
% without further effort on your part as the last section in
% the body of your article BEFORE References or any Appendices.

\numberofauthors{1} %  in this sample file, there are a *total*
% of EIGHT authors. SIX appear on the 'first-page' (for formatting
% reasons) and the remaining two appear in the \additionalauthors section.
%
\author{
% You can go ahead and credit any number of authors here,
% e.g. one 'row of three' or two rows (consisting of one row of three
% and a second row of one, two or three).
%
% The command \alignauthor (no curly braces needed) should
% precede each author name, affiliation/snail-mail address and
% e-mail address. Additionally, tag each line of
% affiliation/address with \affaddr, and tag the
% e-mail address with \email.
%
% 1st. author
\alignauthor
Authors\\
       \affaddr{Web and Internet Science Group}\\
       \affaddr{University of Southampton}\\
       \affaddr{Southampton, UK}\\
       \email{\{a,b,c,d,wh,nrs\}@ecs.soton.ac.uk}
}
% There's nothing stopping you putting the seventh, eighth, etc.
% author on the opening page (as the 'third row') but we ask,
% for aesthetic reasons that you place these 'additional authors'
% in the \additional authors block, viz.
%\additionalauthors{Additional authors: John Smith (The Th{\o}rv{\"a}ld Group,
%email: {\texttt{jsmith@affiliation.org}}) and Julius P.~Kumquat
%(The Kumquat Consortium, email: {\texttt{jpkumquat@consortium.net}}).}
%\date{30 July 1999}
% Just remember to make sure that the TOTAL number of authors
% is the number that will appear on the first page PLUS the
% number that will appear in the \additionalauthors section.

\maketitle
\begin{abstract}

Health social machines ...

\end{abstract}

% A category with the (minimum) three required fields
%\category{H.4}{Information Systems Applications}{Miscellaneous}
%A category including the fourth, optional field follows...
%\category{D.2.8}{Software Engineering}{Metrics}[complexity measures, performance measures]

%\terms{Theory}

%\keywords{ACM proceedings, \LaTeX, text tagging}

\section{Introduction}

Health and well-being are visible indicators of technological
progress, as advances in healthcare and medicine are invariably
reflected in increases in average lifespan, reduction of disease and
suffering, and shortening of time needed to recover from illness and
injury.  As such, it is natural to ask how and whether the Internet
and the Web, two of the most significant inventions in recent human
history, have or may have an effect on health and wellbeing.

In this position paper, we examine a specific class of systems enabled
by the Web and pervasive Internet-enabled systems, which we call
\emph{health social machines}.  We define health social machines to
encompass a broad class of systems that provide
technologically-mediated interaction of large groups of individuals,
typically via a website, app, and sensor-based online community.
Individuals usually communicate and interact, directly or indirectly,
through some mediated or moderation mechanisms, in order to
collectively accomplish or address a health-related problem or need.
Such problems, as we illustrate through examples we provide later, may
be on the scale of an individual's disease or well-being management,
to that of contributing evidence and insight to fundamental questions
at the frontier of modern medicine.

We first describe the emerging landscape of health-related social
machines, identifying sets of classes and characteristics such
machines typically exhibit.  We then focus on specific challenges
faced by these classes in the longer term, and how emerging insights
from behavioural ecnonomics and technological platforms may address
some of these needs.

\section{Current Health Social Machines: A Brief Classificatory Analysis}

We first collected examples of popular health social machines through
a iterative process which started with filtering several popular blogs
focused on health-technology, the ``quantified-self'' and ``life
hacking'', for announcements related to apps and websites dedicated to
addressing health issues. We then clustered the collected candidates
using a Grounded Theory approach.  This process yielded three,
partially overlapping clusters of machines by the ways these machines
sought to address health needs.  Table \ref{table:clusters} these
clusters, comprising \emph{behavioural intervention}, \emph{disease
  management}, and \emph{collective sensemaking} of symptoms, and
associated machines falling in each category.

\subsection{Behavioural Intervention}

The first which we refer to as \emph{behavioural intervention
  machines} are systems that seek to help individuals achieve certain
health related-goals by altering their daily routine(s) and
activities.  The majority of systems we found in this category, which,
itself is the largest of the three categories, aim to help individuals
increase their general activity levels to increase fitness.  Since
these systems generally do not target any particular demographics or
those conditions, we consider them general, preventative health
machines with a focus on increasing fitness.

A large number, but not all, of such fitness machines either require,
or are designed to complement, sensor devices that are intended to
simplify regular measurement of various vital statistics of the
individual.  As such, they are designed to be quick and easy to use,
and, even, in some cases, worn directly on the body, for the measurement
of physiological signals or activity levels, at high temporal
granularity. These on-body activity measurement devices range from
simple accelerometer-based devices (such as the FitBit, Nike
FuelBand), that can approximately estimate the number of
steps/distance the wearer has travelled in a day, to slightly more
complex on-body devices (such as the BodyMedia CORE) that measure
multiple physiological signals in tandem with activity level.  Other,
non-worn devices include iPhone-enabled blood pressure cuffs (e.g.
Withings' Blood Pressure Monitor), internet-connectivity enabled
weight/body mass index scales (e.g., Withings' WiFi Scale), and
iPhone-enabled heart rate, blood oxygen level measuring devices (e.g.,
Zensorium Tinke).  

\subsection{Disease management}

A second class of health social machines aim to help individuals cope
with various kinds of conditions, including illness, disease, and
mental health.  While a few of such systems are general and designed
to accommodate a wide variety of conditions, most of the machines
available today are designed specific to a particular disease or class
of diseases, such as diabetes, depression, Alzheimer's, autism,
post-traumatic stress disorder (PTSD), Coeliac's disease, and so
forth.

\subsection{Collective sensemaking}

The final class of social machines, of which we only found one extant
example, PatientsLikeMe \footnote{PatientsLikeMe -
  \url{www.patientslikeme.com}}, aim to crowd-source knowledge about
disease, symptoms and treatments to individuals who have personally
experienced them.  To do this, PatientsLikeMe facilitates the
independent report of symptoms that individuals are experiencing,
connections between the symptoms and the particular
disease(s)/conditions which they have bene diagnosed with, and the
effects of particular treatments on their conditions.  The result of
this aggregation, at large scale, is a model relating symptoms to
diseases to treatments and effects.  This model can then be used by
other individuals in a number of ways; first, those who are
experiencing symptoms can diagnose themselves based on the
symptom-disease associations, while those with already diagnosed with
a condition can use the disease-treatment-effects model to choose
treatment(s) might the most favourable outcomes and experiences of
others like them.

\begin{table}[htb]
\begin{center}
\begin{tabular}{|p{8cm}|}
\hline
{\bf Preventative wellness} \\
\hline
Device-based: Nike+, FitBit, Withings, BodyMedia, Zeo \\
App-based: RunKeeper, Lose-It, Fitness Pro, GymGoal \\
Site-based: Fitocracy, Traineo, Dailyburn, ExtraPounds, SparkPeople  \\
\hline
{\bf Disease management} \\
\hline
ALZConnected (Alzheimer's patients),  \\
Prevent (Pre-diabetics), \\
BigWhiteWall \\
\hline
{\bf Collective sensemaking} \\
PatientsLikeMe \\
\hline
\end{tabular}
\end{center}
\caption{Consolidated constructs of social machines.} \label{table:clusters}
\end{table}

\section{Current Methods of Support}

\begin{figure*}[htb]
\begin{center}
\includegraphics[width=16cm]{img/table2-summary.png}
\caption{Summary of the kinds of support offered by each of the classes of social machines described in Section 2.} \label{fig:summaryofsupport}
\end{center}
\end{figure*}


In order to understand better how the social machines just described
functioned to help individuals with their health related goals, we
performed an analysis of each site/app's features and derived a set of
observations pertaining to how they supported the goals each sought to
achieve.  We describe each, in turn, next

\subsection{Supporting behaviour change}

In order to support individuals to conform to their behaviour change
interventions, we identified the following elements that these machines
support.

\begin{enumerate}
\item \emph{Measurement and Tracking} - As described earlier, in order
  to be able to provide feedback on progress, most of the machines
  provided support for some degree of data collection, ranging from
  facilitating manual data recording to automatically sensing activity
  and physiological signals through wearable sensors.
\item \emph{Salience and Feedback} - To remind individuals to comply
  with their intervention and reinforce encouragement for incremental
  progress, individuals' progress was made highly salient using a
  number of mechanisms.  For wearable sensors, visible indicators
  (lights/displays) on the sensor often indicated progress, while for
  apps and services, visual prompts, messages and alerts delivered
  through social networks, e-mail, text messages, and asynchronous
  ``push'' notifications were common.
\item \emph{Gamification (Achievements and Prizes)} - To futher
  motivate compliance, many of these systems incorporated a number of
  ``gamification'' features \cite{Deterding:2011:GUG:1979742.1979575}
  meant to make progres seem like play.  Such features typically
  involved rewarding participants with ``points'', ``badges'' and
  ``prizes'' for achieving milestones.
\item \emph{Social encouragement} - In addition to the individual
  gamification elements, social features were provided for most
  machines that encouraged indivdiuals to either compete to achieve
  their objectives either individually or in groups, or to support one
  another by ``cheering them on'' and supporting them in various ways.
  Competitive elements included ``battles'' and ``challenges'', supportive
  capabilities included ``cheerleading'', wagering, and donating ``points''
  in support of another individual's cause.
\end{enumerate}

\subsection{Facilitating disease management}
Disease management machines provide three kinds of support to
individuals; first, like the behavioural intervention machines, to
deliver actual interventions specific to individuals' conditions.  One
of the best examples of such intervention delivery is BigWhiteWall,
which delivers mental health services through an online social network
through a full-time staff of professional counselors who monitor the
site 24 hours a day.

The second role these marhines serve is a place to exchange knowledge
and insight, serving as both \emph{answer gardens} \cite{answergarden}
and \emph{serendipitous knowledge archives} \cite{knowledgearchive}.
As answer gardens, these sites let individuals find and post answers
to specific questions they have.  As knowledge archives, individuals
with similar circumstances can post and more easily stumble upon tips
that are relevant to their particular situation(s) - which might help
them improve their situation (even if they did not know to
specifically ask or look for this information to begin with).

Third, these systems provide a mechanism of social emotional support,
both in terms of empathy from people who have experienced similar
situations in the past, or sympathy, from those who can relate and
provide words of encouragement or advice.

\subsection{Enabling crowd-based sensemaking}

Health machines that seek to crowdsource information about disease and
treatments at large scale require the ability to acquire information
effectively and as accurately as possible from participants.  Thus,
effective elicitation of information becomes a primary challenge.
Towards this capabillity, PatientsLikeMe supported a structured
elicitation approach for the gathering of symptoms, relevant diagnosed
diseases, treatments, and reports of experiences.  Gathering
structured data directly (in terms of ratings, diseases and treatments
from a fixed lexicon) allowed these data can be compared and
aggregated automatically across individuals.  

Beyond elicitation, such sites require the ability to produce useful
views of collected data so that participnats are not overwhelmed by
the volumes of raw data produced by others.  Towards this end,
PatientsLikeMe used the structured data captured to synthesize raw
simple aggregate visualisations and result summaries that could be
easily interpreted.

\section{Challenges and Limitations}

In this section we synthesise a set of problems and challenges that
have been voiced concerning the effectiveness of health social
machines, including concerns voiced by the medical community. 

\subsection{Potential Dangers of Self-Diagnosis}

Many of the examined machines encourage individuals to make their own
decisions concerning their health, with such slogans as ``take back
your health now!''.  To do this, they give seemingly appropriate and
relevant information to individuals that let them choose options, such
as intervention programmes, or, in the case of PatientsLikeMe,
treatments from which to choose.  One of the appeals of this idea is
that if individuals are the ones that choose their intervention, they
would feel more ownership over it and would be more likely to comply
fully.

However, clinicians and medical professionals have voiced concern
about this intiiatve, because individuals lacking medical expertise or
experience are likely to make bad decisions on incomplete knowledge
that may put their health in peril.  For example, an individual who is
feeling unwell might suspect that they need more physical exercise and
sign up to an ``increased activity'' intervention programme, when in
fact they might have a heart condition that might become more severe
under increased cardiovascular strain.  If they had seen a medical
professional instead of self-diagnosing themselves, the heart
condition may have been identified earlier and more easily addressed.

\subsection{Quality and Adherence to Intervention Programmes}

A second associated concern pertaining to democratising the creation
and administration of intervention programmes is simply, first, that
having non-medical professionals devise intervention programmes means
these programmes have not been rigorously evaluated, and thus may be
less effective (or entirely ineffective) beyond a placebo effect
\cite{placeboeffect}.  

Furthermore, professional clinicians, therapists and psychiatrists
have methods to make sure individuals are comply and adhere with their
intervention and receiving maximum benefit.  When the professional is
removed from the loop, interventions may become less effective as
patients are not guided to adhere to the prescribed programmes. Thus
other mechanisms (such as the gamification components described
above) may need to take this role.  

On the other hand, the potential for the new wearbale sensor activity
monitors means that an indivdiual's activities can be recorded at
little or no cost, and analysed to produce a more complete picture of
an individual's activities; this could be useful in increasing
compliance and understanding of how an individual is behaving and can
improve their performance.

\subsection{Sustaining Motivation}

A second challenge concerns the effectiveness the methods applied by
health social machines towards sustaining long-term involvement, in
particular for encouraging compliance and adherence to the more
difficult intevenions that challenge the very centres of people's
motivational systems, such as those concerning weight loss and
mental health.

In particular, a number of recent studies on gamification have
revealed that simple approaches for introducing extrinsic reward, such
as points and ``badges'' may wear off after a short initial period of
novelty \cite{therebedragons}.  Even the most successful example of
gamification, Foursquare, experienced a widespread engagement problem
across its user population six to 12 months after adoption
\cite{browningattheedges}. Meanwhile, other studies of gamification
showed that gamification elements actually decreased participation by
reducing individuals' intrinsic motivation to participate, which, in
some cases returned when the gameificaiton elements were once again
removed \cite{Thom:2012:RGE:2145204.2145362}.

\subsection{Self-Report, Bias and Explaining-Away}

In the realm of crowd-sourcing disease knowledge, several of the
health machines rely on self-report as the primary method of knowledge
elicitation.  Controlled studies have demonstrated the many problems
of self-report across domains, with the most significant biases being
revealed in health, such as concerning a person's estimates of their
own fitness, including body mass and weight \cite{elgar2005validity}
and happiness, including depression \cite{hunt2003self}. Such biases,
which vary among individuals and factors estimated, could
significantly impact the validity of the data collected if it is
information concerning the diseases and symptoms.

Of additional difficulty concerning self-report arises in situations
where there is need for patients to perform causal inference between
symptoms, causes and treatments, such the case with PatientsLikeMe,
which asks patients to describe symptoms experienced with a disease
and the outcome of particular treatments.  The well-studied
psychological effect of confirmation bias \cite{confirmationbias},
which causes individuals to gather evidence in support of pre-existing
beliefs, could cause individuals to report what they \emph{think they
  should see} instead of what they actually experienced.  Furthermore,
illusory correlation \cite{chapman1969illusory} could cause
individuals to draw connections between experiences, situations or
conditions merely due to their salience or co-occurrence, rather than
due to an actual causal relationship. A particular type of illusory
correlation is \emph{explaining away}, in which individuals attribute
causes to the most salient explanation, rather than the most probable
one \cite{gilovich1983biased}.

An additional bias that emerges when self-reports are produced in
groups is \emph{collective conservatism}, a very strong bias that
people in a group tend to say (or agree with) what others say instead
of contributing what they actually observe, feel or know.  This
phenomenon, well studied in social psychology
(e.g. \cite{aronson2003readings}), has been shown to cause individuals
to conform to, and even believe, incorrect group conclusions even when
they disagree with these conclusions themselves.  In public discussion
forums such as PatientLikeMe, collective conformation could dramatically
shape the conclusions that patients arrive at, and that future patients
might experience.

\section{Towards More Effective Machines}

% \section{Approaches}
%% \subsection{Social pressure and Motivation: Gym Memberships and Personal Trainers}
%% \subsection{Present: Channel factors, access, convenience}
%% \subsection{Present: Salience and reminders}
%% \subsection{Futures: Personalised Activity Diaries}
%% \subsection{Futures: Citizen-medicine}

\section{Acknowledgments}

This work is supported under SOCIAM: The Theory and Practice of Social
Machines.  The SOCIAM Project is funded by the UK Engineering and
Physical Sciences Research Council (EPSRC) under grant number
EP/J017728/1 and comprises the Universities of Southampton, Oxford and
Edinburgh.

%
% The following two commands are all you need in the
% initial runs of your .tex file to
% produce the bibliography for the citations in your paper.
\bibliographystyle{abbrv}
\bibliography{health-social-machines}  % sigproc.bib is the name of the Bibliography in this case
% You must have a proper ".bib" file
%  and remember to run:
% latex bibtex latex latex
% to resolve all references
%
% ACM needs 'a single self-contained file'!
%
%APPENDICES are optional


\balancecolumns % GM June 2007
% That's all folks!
\end{document}
